\usepackage{array,graphicx,hyperref,geometry,fancyhdr,
setspace,xeCJK,fontspec,xunicode,amsmath,booktabs,diagbox,
,syntonly,multicol,balance,tocloft,titletoc,chngcntr,ifthen,indentfirst}

%使图片能按照章节自动编号,如图2.1,图2.2 ...
\counterwithin{figure}{section}
\counterwithin{table}{section}

%\syntaxonly  %取消注释后只检查语法,不输出pdf文档

%设置正文的英文字体族
\setmainfont{Times New Roman} 
\setsansfont{Consolas}
\setmonofont{Courier New}

%设置正文的中文字体族
\setCJKmainfont{SimSun}
\setCJKsansfont{KaiTi}
\setCJKmonofont{Fangsong}

%基础字体设置
% \setCJKmainfont{SimSun}[AutoFakeBold={2.17},ItalicFont=KaiTi] %实现宋体加粗
\setCJKfamilyfont{song}{SimSun}           %宋体 song
\newcommand{\song}{\CJKfamily{song}}
\setCJKfamilyfont{fs}{FangSong}           %仿宋2312 fs
\newcommand{\fs}{\CJKfamily{fs}}
\setCJKfamilyfont{yh}{Microsoft YaHei}    %微软雅黑 yh
\newcommand{\yh}{\CJKfamily{yh}}
\setCJKfamilyfont{hei}{SimHei}            %黑体  hei
\newcommand{\hei}{\CJKfamily{hei}}
\setCJKfamilyfont{kai}{KaiTi}             %黑体  hei
\newcommand{\kai}{\CJKfamily{kai}}

%基础字号设置
\newcommand{\chuhao}{\fontsize{42pt}{\baselineskip}\selectfont}
\newcommand{\xiaochu}{\fontsize{35pt}{\baselineskip}\selectfont}
\newcommand{\yihao}{\fontsize{26.4pt}{\baselineskip}\selectfont}
\newcommand{\erhao}{\fontsize{22pt}{\baselineskip}\selectfont}
\newcommand{\xiaoer}{\fontsize{18pt}{\baselineskip}\selectfont}
\newcommand{\sanhao}{\fontsize{15.75pt}{\baselineskip}\selectfont}
\newcommand{\sihao}{\fontsize{14pt}{\baselineskip}\selectfont}
\newcommand{\xiaosi}{\fontsize{12pt}{\baselineskip}\selectfont}
\newcommand{\wuhao}{\fontsize{10.5pt}{\baselineskip}\selectfont}
\newcommand{\xiaowu}{\fontsize{9pt}{\baselineskip}\selectfont}
\newcommand{\liuhao}{\fontsize{7.875pt}{\baselineskip}\selectfont}
\newcommand{\qihao}{\fontsize{5.25pt}{\baselineskip}\selectfont}

%这两行代码使得宋体能够正常地加粗
\setCJKmainfont{SimSun}[AutoFakeBold={2},ItalicFont=KaiTi] %实现宋体加粗

\setCJKfamilyfont{song}[AutoFakeBold = {2}]{SimSun}
\renewcommand*{\song}{\CJKfamily{song}}


\newcommand{\n}{\par}         %换行(缩进值固定)
\newcommand{\kg}{\quad}       %空格     
\newcommand{\konghang}{\qquad\par} %空行

\geometry{a4paper,top=3.2cm,bottom=2.8cm,right=2.25cm,left=2.8cm}
\ziju{0.000 1pt}


\newcommand{\oddHeader}{}
\newcommand{\evenHeader}{}

\newcommand{\setOddPageHeader}[1]{\renewcommand{\oddHeader}{#1}}
\newcommand{\setEvenPageHeader}[1]{\renewcommand{\evenHeader}{#1}}

\lhead{} 
\newcommand{\mkHeader}{\chead{\ifthenelse{\isodd{\thepage}}{\oddHeader}{\evenHeader}} }
\rhead{}
\lfoot{} 
\cfoot{-\thepage-} 
\rfoot{}
\renewcommand\headrulewidth{0pt} %取消页眉横线

%双线页眉                                              
\makeatletter
\def\headrule{{\if@fancyplain\let\headrulewidth\plainheadrulewidth\fi%
\hrule\@height0.5pt         %上页眉横线粗细设置
\@width\headwidth\vskip1pt  %两条页眉横线之间的垂直间距
\hrule\@height 0.5pt\@width\headwidth %下页眉横线粗细设置
\vskip-2\headrulewidth\vskip12.5pt} %双线与下面正文之间的垂直间距1pt
} 
\makeatother
\pagestyle{fancy}


\newcommand{\artFormatSetup}{
\xiaosi
\setcounter{page}{1}   %页码编号从第一页开始
\pagenumbering{arabic} %使用阿拉伯数字
\pagestyle{fancy}
%linespread 和 baselineskip相配合即可设置行距,当前为20磅行距
\linespread{1.25}
\setlength{\baselineskip}{20pt} 
\setlength{\parindent}{2em}
}


%各级标题格式设置
\usepackage[raggedright,compact]{titlesec}  %标题左对齐
\titleformat{\section}{\sanhao\song\bfseries}{\thesection}{6pt}{}
\titleformat{\subsection}{\sihao\song\bfseries}{\thesubsection}{6pt}{}
\titleformat{\subsubsection}{\xiaosi\song\bfseries}{\thesubsubsection}{6pt}{}

%设置各级标题的段前段后垂直距离
\titlespacing*{\section}{0pt}{10pt}{17pt}
\titlespacing*{\subsection}{0pt}{0pt}{8.2pt}
\titlespacing*{\subsubsection}{0pt}{-5pt}{0pt}

\newcommand{\yiji}[1]{\section{#1}}
\newcommand{\erji}[1]{\subsection{#1}}
\newcommand{\sanji}[1]{\subsubsection{#1}}


%目录格式设置
\renewcommand{\cftdotsep}{1} %缩小目录引导线的点距
%使一级标题带目录引导线
\renewcommand{\cftsecdotsep}{\cftdotsep}
\renewcommand{\cftsecleader}{\cftdotfill{\cftsecdotsep}}
%各级标题字体设置
\renewcommand{\cftsecfont}{\zihao{4}\songti}
\renewcommand{\cftsubsecfont}{\zihao{4}\songti}
\renewcommand{\cftsubsubsecfont}{\zihao{4}\songti}
%各级标题对应的页码字体设置
\renewcommand{\cftsecpagefont}{\xiaosi}
\renewcommand{\cftsubsecpagefont}{\xiaosi}
\renewcommand{\cftsubsubsecpagefont}{\xiaosi}
%各级标题垂直间距
\renewcommand{\cftbeforesecskip}{10pt}
\renewcommand{\cftbeforesubsecskip}{10pt}
\renewcommand{\cftbeforesubsubsecskip}{10pt}
%连线和页码之间的间距
\contentsmargin{5pt}
\hypersetup{colorlinks=true,linkcolor=black} %去除目录标题的红框




\usepackage{listings}
\usepackage{xcolor}
\lstset{
columns=flexible, %优化非等宽字体显示效果
numbers=left,     %行号左侧显示
numberstyle= \rm, %行号字体设置
basicstyle=\fontspec{Consolas},   %代码主字体设置(关键词,标识符等字体可另外设置)
xleftmargin = 18pt, %行号与左侧页边距的距离
frame = shadowbox,
tabsize = 4
} 

\newcommand{\authorinfo}[2]{\centerline{\kai\sihao{$\bullet$ #1\quad $\bullet$ #2}}}
\newcommand{\subtitle}[1]{\fs\sihao\raggedleft{ -----\,#1\hspace{2cm}\mbox{}}}
\renewcommand{\title}[1]{\hei\erhao{#1}\par}
\renewcommand{\contentsname}{\centerline{目 \quad 录}}

%图片设置
\usepackage[font=rm,labelsep=quad]{caption} 
%插入图片宏,其中默认值0.5赋值给#1,若修改这个默认值:
%\tupian[新的值]{arg2}{arg3}
\newcommand{\img}[3][0.5]{
\begin{figure}[h]
\begin{center}
\includegraphics[width = #1\textwidth]{#2}
\caption{#3}
\end{center}
\vspace{-0.5cm}
\end{figure}
}

%设置位于图片下方的图题与上方图片和下方正文的间距
\setlength{\abovecaptionskip}{2pt}
\setlength{\belowcaptionskip}{0pt}

\newenvironment{biao}[2][1.5]
{\begin{table}[htbp]
\centering
\renewcommand{\arraystretch}{#1} %设置表格中的行高,使文本与表格上横线有一定距离
}
{\end{tabular}
 \end{table}}

%三线表的上中下三条线样式控制
\newcommand{\topline}{\toprule[1.75pt]}
\newcommand{\midline}{\midrule[0.8pt]}
\newcommand{\btmline}{\bottomrule[1.75pt]}

\newcommand{\biaoge}[3][0.8]
{
\begin{table}[htbp]
\centering
%\arraystretch设置表格中的行高,使文本与表格上横线有一定距离
\renewcommand{\arraystretch}{2} 
%\resizebox设置表格(实际上是一个box)的宽度,!表示保持宽高比
\caption{#2}
\resizebox{#1\textwidth}{!}{\xiaosi#3} 
\end{table}
}


\newcommand{\mkMuLu}{
\tableofcontents
\clearpage} %生成目录

%生成指定列的表格
\newcounter{cols}
\newcommand{\mkTable}[4][c]{
\setcounter{cols}{#3}
\begin{table}[!ht]
\caption{#2}\label{tab:parametervalues}
\begin{tabular*}{\hsize}{@{}@{\extracolsep{\fill}}*\thecols{#1}@{}}
#4
\end{tabular*}
\end{table}
}


\newcommand{\cnAbsTitle}[1]{
\begin{center}
\setlength{\baselineskip}{20pt}
\xiaoer{\song{\textbf{#1}}}\\
\sanhao{\song{\textbf{摘\ 要}}}
\end{center}	
\n
\addcontentsline{toc}{section}{摘要}
}


\newcommand{\cnKeywords}[1]{
\\\\\textbf{\song 关键词:}#1\clearpage
}

%创建英文摘要标题
\newcommand{\enAbsTitle}[1]{
\begin{center}
\setlength{\baselineskip}{20pt}
\xiaoer{\song{\textbf{#1}}}\\
\setlength{\baselineskip}{31pt}
\sanhao{\song{\textbf{ABSTRACT}}}
\end{center}    
\n
\addcontentsline{toc}{section}{Abstract}
}
%创建英文摘要关键词
\newcommand{\enKeywords}[1]{
\\\\\textbf{\song KEY \, WORDS:\quad}#1\n
}
\newcommand{\Title}{文章标题}
\newcommand{\Author}{许萌}
\newcommand{\studentID}{1234567}
\newcommand{\class}{计本1703}
\newcommand{\major}{计算机科学与技术本科}
\newcommand{\instructor}{许萌}

\newcommand{\setTitle}[1]{\renewcommand{\Title}{#1}}
\newcommand{\setAuthor}[1]{\renewcommand{\Author}{#1}}
\newcommand{\setStudentID}[1]{\renewcommand{\studentID}{#1}}
\newcommand{\setClass}[1]{\renewcommand{\class}{#1}}
\newcommand{\setMajor}[1]{\renewcommand{\major}{#1}}
\newcommand{\setInstructor}[1]{\renewcommand{\instructor}{#1}}



\newcommand{\mkCover}{
\begin{center}
% \parbox{15.9cm}{论文分类号:2 \hfill 学校代码: 10708\kg \kg \n \hfill 学\phantom{占位}号: 1306020}
\begin{figure}[!h]
\raggedright
\vspace{-10pt}
\includegraphics[width=7.5cm]{__pic//logo.png}
\end{figure}
\n
\centering
\vspace{72pt}
\textbf{\xiaochu{集\hspace{17pt}中\hspace{17pt}性\hspace{17pt}综\hspace{17pt}合\hspace{17pt}实\hspace{17pt}训\hspace{17pt}报\hspace{17pt}告}\n}
\vspace{40pt}
% \textbf{\yihao{Thesis\hspace{5pt}for\hspace{5pt}Bachelor's\hspace{5pt}Degree}\n \xiaochu{\phantom{占位}\n}}
\vspace{16pt}
\textbf{\setlength\parskip{7pt} \erhao{\Title}}
\parbox{10cm}{
\vspace{40pt}
\song\sanhao
\linespread{1.5}
\setlength{\baselineskip}{31pt} 
\centering
\Author \n
\vspace{70pt}
\raggedright

学\phantom{\,\,\,\,占\,\,\,\,位\,\,\,\,}号:\studentID \n
班\phantom{\,\,\,\,占\,\,\,\,位\,\,\,\,}级:\class \n
指\,\,\,\,导\,\,\,\,教\,\,\,\,师:\instructor \n
专\,\,\,\,业\,\,\,\,名\,\,\,\,称:\major \n
}
\thispagestyle{empty}
\clearpage
\end{center}
}


% \bibliographystyle{plain}

